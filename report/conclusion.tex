\section{Conclusion}

In the soft X-ray domain explored by Chandra, Venus was a clear source. In a harder domain (3 keV - 10 keV), this is less obvious with the data presented here. Many differences exist between both analyses, physical and data wise. First of all, Venus doesn't have strong fluorescent scattering in a > 3 keV energy range because these fluorescence mainly concern heavier elements that are only present in the Venusian atmosphere as traces\footnote{See \url{https://physics.nist.gov/PhysRefData/XrayTrans/Html/search.html} for a list of fluorescence scattering energies and \cite{Futaana2017SolarAtmosphere}}. Secondly, Venus was only serendipitously observed in the \textit{INTEGRAL} data and it moved in the image. The data was therefore by construction less quality as the one retrieved by the Chandra spacecraft although the observation conditions were almost as optimal given the orbital parameters. However, the average fluxes calculated were all positive at a high significance level which suggest that the observation was not unsuccessful although more analysis should be made to ascertain this statement. Probing these higher energies are also the occasion to explore different emission processes and thresholds should be computed. Indeed, charge exchange interactions between highly charged heavy solar wind ions and atmospheric neutrals is the dominant process for the X-ray emission of comets for which Venus is a giant version\cite{Futaana2017SolarAtmosphere}.

The concordance between the solar and venusian flux showed its limits of applicability because of the lack of information on potential events hitting Venus. The event direction locator using the HEK gives however a good idea of the high number of events hitting the magnetic field deprived planet every single day.

In conclusion, the study of solar system planets in harder X-ray energies presents an interesting avenue for exploration. The extensive data gathered by the \textit{INTEGRAL} satellite over the past 19 years shows promise for conducting further analysis in this field. By delving deeper into this research, we have the potential to gain valuable insights into the behaviour of our neighbouring planets. The available \textit{INTEGRAL} data offers a solid foundation for future studies and encourages us to continue pushing the boundaries of this analysis.

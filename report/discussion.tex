\section{Discussion}
%talk about fluorescence maybe
%talk about the next steps, what's to do: check the abundances or check with models. Cite some articles(check the word document). See how many photons would be expected. 
% would be cool to put limits on processes: charge-exchange, fluorescence of heavy elements etc...
%say that the second method is probably the most correct one
%talk about the significance tests.
% compare with literature

The events are all directed towards Earth and shows the FoV limitation of LASCO and GOES to Earth directed events only. Venus sees a rotated side of the Sun by about $\sim 45$°. But the number of events (54 CMEs and 130 flares $\geq$ C) detected in the time interval from the 19.04.2022 to the 24.04.2022 is high. There are therefore no reason to think that the Sun's activity wasn't as high on the Venus' directed side. The fluxes retrieved here should only be used as general indicators of the Sun's activity not as general events directly impacting Venus although the events, in particular the most powerful ones, often have spatial extents of 180° and therefore impacting both planets at the same time.
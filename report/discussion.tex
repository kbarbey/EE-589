\section{Discussion}
%talk about fluorescence maybe
%talk about the next steps, what's to do: check the abundances or check with models. Cite some articles(check the word document). See how many photons would be expected. 
% would be cool to put limits on processes: charge-exchange, fluorescence of heavy elements etc...
%say that the second method is probably the most correct one
%talk about the significance tests.
% compare with literature

%say that both assumptions have argument for and against them but not constant is probably better because of variability.

The fluxes retrieved from Venus' position all show similar averages. The PSF modelling is however the better way to retrieve Venus' flux as it is a somewhat punctual source with very limited extent in the image given its apparent size and the detector's resolution. The question still arises though whether the non-constant or constant model is the best one. Since the solar flux is the main influencer of the X-ray emission of the planet and since it varies on the time scale of minutes, the non-constant model would appear to be the best one especially in high activity periods of the Sun.

Compared to \cite{Dennerl2002DiscoveryChandra}, the order of magnitude of the flux found is the same. However, the expected behaviour would be a decrease in the average flux and not an increase as \textbf{Fig.} \ref{comp_spec} suggests. The \textit{Dennerl} paper uses only fluorescence as a model and this could be accounted for the discrepancy. The next step of the analysis reported here would be to put thresholds on the different possible emission processes such as charge exchange which was modelled for the Venus atmosphere in \cite{Gombosi1981THEABSORPTION}.

The KS tests performed on the vertical distributions of the fluxes all reject the idea of a direct correlation between the solar flux variability and Venus' although it is less clear for the 24.04.2022 window(4.8\%). This is the same conclusion that arised from \cite{Dennerl2002DiscoveryChandra} and the same explanation can be made: Venus was seeing another portion of the Sun and the solar flux variability detected were localised and didn't necessarily impact Venus. The number of points is low though and increasing the statistics would be the next step.

The events are all directed towards Earth and shows the FoV limitation of LASCO and GOES to Earth directed events only. Venus sees a rotated side of the Sun by about $\sim 45$°. But the number of events (54 CMEs and 130 flares $\geq$ C) detected in the time interval from the 19.04.2022 to the 24.04.2022 is high. There are therefore no reason to think that the Sun's activity wasn't as high on the Venus' directed side. The fluxes retrieved here should only be used as general indicators of the Sun's activity not as general events directly impacting Venus although the events, in particular the most powerful ones, often have spatial extents of 180° and therefore impacting both planets at the same time. Locating the direction of events not directed towards Earth requires either an X-ray spacecraft to be orbiting the Sun at the desired position or using other methods such as triangulation \cite{Liu2010GeometricAU} with the STEREO spacecrafts in heliocentric orbit. This was out of the scope of this project and not explored. The \cite{Dennerl2002DiscoveryChandra} study had the same trouble as they used the GOES missions solar flux data.